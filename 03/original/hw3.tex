\documentclass[11pt]{article}
\usepackage[margin=0.8in]{geometry}
\usepackage{amsmath}
\usepackage[boxed]{algorithm2e}
\usepackage[dvipsnames]{xcolor}
\usepackage{graphicx}
\usepackage{algorithm}
\usepackage{algorithmic}
\usepackage{hyperref}
\usepackage{tikz,graphics,color,fullpage,float,epsf,caption,subcaption}
​
​
\title{CS3510-A: Design and Analysis of Algorithms, Spring 2021} 
​
\author{Homework-3}
​
\begin{document}
% \maketitle
\begin{center}
    
    \LARGE CS3510-A: Design and Analysis of Algorithms, Spring 2021 \\ \vspace{1em} 
    \large Homework-3 \\ \vspace{0.5em}
    February 2, 2021
\end{center}
\thispagestyle{empty}
\pagestyle{empty}
​
\noindent
\begin{center}
{\bf DUE DATE: Tuesday, February 9, 1l:59pm}
\end{center}
​
\noindent
{\bf Note-1:} Your homework solutions should be electronically formatted as a single PDF document that you will upload on Gradescope. 
If you have to include some handwritten parts, please make sure that they are very clearly written and that you include them as high resolution images. \\
​
\noindent
{\bf Note-2:} Please think twice before you copy a solution from another student or resource (book, web site, etc). 
It is not worth the risk and embarrassment. \\
​
\noindent
{\bf Note-3:} You need to {\bf explain/justify} your answers. Do not expect full credit if you just state the correct answer. \\
​
\noindent
{\bf Note-4: You will get 2 extra points if you submit electronically typed solutions instead of hand-written.} 
​
\newpage
\section*{Problem-1 (30 points)}
You are given a DAG. How would you check in linear time if the DAG includes a Hamiltonian path, i.e., a (directed) path that traverses every vertex exactly once? \\ \\\textbf{Note 1:} Please provide a detailed description of your solution in plain English or in pseudocode. \\
\textbf{Note 2:} You are also allowed to use BFS and DFS as blackbox algorithms (you do not need to explain how it works if you use it) so long as you detail the inputs and outputs to the algorithms as well as any modifications you make.​


\subsection*{Solution}



\newpage
\section*{Problem-2 (35 points)}
You are given a directed graph in which each node $u \in V$ has an associated capacity $c_u$ which is a positive integer. Define the array {\em bottleneck} as follows: for each $u \in V$,
\begin{center}
    $bottleneck[u] =$ capacity of the lowest-capacity node reachable from $u$ (including $u$ itself).
\end{center}
\noindent
For instance, in the graph below (with capacities shown inside each vertex), the bottleneck values of all nodes are equal to 1.
Your goal is to design an algorithm that fills in the entire bottleneck array (i.e., for all vertices). \\
\begin{center}
    \begin{tikzpicture}
  [scale=.8,auto=left,very thick, every node/.style={circle,fill=blue!20}]
  \node (n6) at (1,6) {A (6)};
  \node (n4) at (4,8)  {B (4)};
  \node (n5) at (8,8)  {C (5)};
  \node (n1) at (11,6) {D (1)};
  \node (n2) at (8,4)  {E (2)};
  \node (n3) at (4,4)  {F (3)};

  \foreach \from/\to in {n6/n3, n6/n4,n4/n2,n3/n5, n4/n5,n5/n1,n2/n1,n2/n3}
    \draw[->] (\from) -- (\to);
\end{tikzpicture}
\end{center}

\begin{enumerate}
    \item[(a)] Design a linear-time algorithm that works for DAGs.
    \item[(b)] Extend your previous answer to a linear-time algorithm that works for all directed graphs.
\end{enumerate}
\\ \\
\textbf{Note 1:} Please provide a detailed description of your solution in plain English or in pseudocode. \\ \\
\textbf{Note 2:} You are also allowed to use BFS and DFS as blackbox algorithms (you do not need to explain how it works if you use it) so long as you detail the inputs and outputs to the algorithms as well as any modifications you make.​
\subsection*{Solution}



​
\newpage
\section*{Problem-3 (35 points)}
\noindent
You're helping some security analysts monitor a collection of networked
computers, tracking the spread of an online virus. There are $n$ computers in the system, labeled $C_1, C_2, ... C_n$, and as input you're given a collection of trace data indicating the times at which pairs of computers communicated. Thus the data is a sequence of ordered triples ($C_i, C_j, t_k$); such a triple indicates that $C_i$ and $C_j$ exchanged bits at time $t_k$. There are $m$ triples total.  \\ \\ \noindent
We'll assume that the triples are presented to you in sorted order of time. For purposes of simplicity, we'll assume that each pair of computers communicates at most once during the interval you're observing. \\ \\ \noindent
The security analysts you're working with would like to be able to answer questions of the following form: If the virus was inserted into computer $C_a$ at time $x$, could it possibly have infected computer $C_b$ by time $y$? The mechanics of infection are simple: if an infected computer $C_i$ communicates with an uninfected computer $C_j$ at time $t_k$ (in other words, if one of the triples ($C_i, C_j$, $t_k$) or ($C_j,C_i, t_k$) appears in the trace data), then computer $C_j$ becomes infected as well, starting at time $t_k$. Infection can thus spread from one machine to another across a sequence of communications, provided that no step in this sequence involves a move backward in time. Thus, for example, if $C_i$ is infected by time $t_k$, and the trace data contains triples ($C_i, C_j$, $t_k$) and ($C_j,C_q, t_r$), where $t_k < t_r$, then $C_q$ will become infected via $C_j$. (Note that it is okay for $t_k$ to be equal to $t_r$; this would mean that $C_j$ had open connections to both $C_i$ and $C_q$ at the same time, and so a virus could move from $C_i$ to $C_q$.)  \\ \\ \noindent
For example, suppose $n= 4$, the trace data consists of the triples:
\begin{center}
$(C1, C2,4), (C2,C4,8), (C3,C4,8), (C1,C4, 12)$

\end{center}
and the virus was inserted into computer $C_1$ at time 2. Then $C_3$ would be infected at time 8 by a sequence of three steps: first $C_2$ becomes infected at time 4, then $C_4$ gets the virus from $C_2$ at time 8, and then $C_3$ gets the virus from $C_4$ at time 8. On the other hand, if the trace data were:
\begin{center}
$(C2, C3, 8), (C1,C4, 12), (C1, C2, 14)$
\end{center}
and again the virus was inserted into computer $C_1$ at time 2, then $C_3$ would not become infected during the period of observation: although $C_2$ becomes infected at time 14, we see that $C_3$ only communicates with $C_2$ before $C_2$ was infected. There is no sequence of communications moving forward in time by which the virus could get from $C_1$ to $C_3$ in this second example. \\ \\ \noindent
Design an algorithm that answers questions of this type: given a collection of trace data, the algorithm should decide whether a virus introduced at computer $C_a$ at time $x$ could have infected computer $C_b$ by time $y$. The algorithm should run in time $O(m+n)$.
\\ \\
\noindent
\textbf{Note 1:} Please provide a detailed description of your solution in plain English or in pseudocode. Also, explain the time complexity and correctness of the prescribed algorithm.\\ \\
\textbf{Note 2:} You are also allowed to use BFS and DFS as blackbox algorithms (you do not need to explain how it works if you use it) so long as you detail the inputs and outputs to the algorithms as well as any modifications you make.​

\subsection*{Solution}

\end{document}